% Latex generates professionally typeset documents. It's different
% from word processors like Microsoft Word in that it separates style from
% content. It lets you write content without worrying too much about things like
% fonts, positioning of tables, and references. Latex has carefully considered
% rules to that sort of that heavy-lifting.

% Latex files are text files where most of the content is the words, sentences,
% and paragraphs. When complete, latex files are fed into the pdflatex program
% to create a pdf - we usually call this compiling the document.

% Within the text you can put comments and commands. These lines are comments.
% On any line, everything after a percent sign (%) is treated as a comment.
% Comments don't have any effect on the output file.

% Commands start with a backslash. The first command is usually \documentclass,
% as below. It tells latex the type of document to generate: book, article,
% letter, etc. Commands can take parameters and/or options.

% Parameters appear in curly braces after the command. Options appear in square
% brackets between the command and the parameters. The \documentclass command
% below has "article" as a parameter and "a4paper" as one option. Multiple
% options are separated by commas.

% Some commands come in pairs like \begin{center} and \end{center}, which is
% usually called an environment. Content in a center environment is centered.

% Latex is built on a language called TeX. TeX is a simple language with only a
% few built-in commands. TeX allows us to create our own commands. LaTeX is a
% huge collection of such programmer-created commands in one handy package.

% There are only a few types of document built-in to latex, like article.
% If you google "latex article options" you'll find lots of sites listing
% what each option does.
\documentclass[12pt, a4paper, oneside, openany, dvipsnames, hidelinks]{article}

% Lot's of latex programmers have created extra packages for Latex.
% They're not enabled by default as they affect the length of time it takes
% to compile the document. They can enable things like colour and hyperlinks
% in the output pdf. The \usepackage command will include a package, so long
% as it is installed along with latex. TeXLive include lots of standard
% packages by default. Google "latex minted" to see what it does.

% Enables the use of colour. 
\usepackage{xcolor}
% Syntax high-lighting for code. Requires Python's pygments.
\usepackage{minted}
% Enables the use of umlauts and other accents.
\usepackage[utf8]{inputenc}
% Diagrams.
\usepackage{tikz}
% Settings for captions, such as sideways captions.
\usepackage{caption}
% Symbols for units, like degrees and ohms.
\usepackage{gensymb}
% Latin modern fonts - better looking than the defaults.
\usepackage{lmodern}
% Allows for columns spanning multiple rows in tables.
\usepackage{multirow}
% Better looking tables, including nicer borders.
\usepackage{booktabs}
% More math symbols.
\usepackage{amssymb}
% More math fonts, like mathbb.
\usepackage{amsfonts}
% More math layouts, equation arrays, etc.
\usepackage{amsmath}
% More theorem environments.
\usepackage{amsthm}
% More column formats for tables.
\usepackage{array}
% Adjust the sizes of box environments.
\usepackage{adjustbox}
% Better looking single quotes in verbatim and minted environments.
\usepackage{upquote}
% Better blank space decisions.
\usepackage{xspace}
% Better looking tikz trees.
\usepackage{forest}
% URLs.
\usepackage{hyperref}
% For plotting.
\usepackage{pgfplots}
% For filler.
\usepackage{lipsum}
% For line spacing
\usepackage{setspace}
% For changing spacing on pages.
\usepackage{geometry}
% For Gantt charts.
\usepackage{pgfgantt}

% The tikz package allows us to create diagrams and plots using latex
% commands. It defines some extra commands, like \usetikzlibrary below.

% Various tikz libraries.
% For drawing mind maps.
\usetikzlibrary{mindmap}
% For adding shadows.
\usetikzlibrary{shadows}
% Extra arrows tips.
\usetikzlibrary{arrows.meta}
% Old arrows.
\usetikzlibrary{arrows}
% Automata.
\usetikzlibrary{automata}
% For more positioning options.
\usetikzlibrary{positioning}
% Creating chains of nodes on a line.
\usetikzlibrary{chains}
% Fitting node to contain set of coordinates.
\usetikzlibrary{fit}
% Extra shapes for drawing.
\usetikzlibrary{shapes}
% For markings on paths.
\usetikzlibrary{decorations.markings}
% For advanced calculations.
\usetikzlibrary{calc}

% No prizes for guessing what the following few commands do.

% GMIT colours.
\definecolor{gmitblue}{RGB}{20,134,225}
\definecolor{gmitred}{RGB}{220,20,60}
\definecolor{gmitgrey}{RGB}{67,67,67}

% Uncomment for one and a half line spacing.
% \onehalfspacing

% Tell minted to use the following colour scheme. 
\usemintedstyle{manni}
% Set some minted options.
\setminted{frame=lines, framesep=2mm, baselinestretch=1.2, linenos}

% The title.
\title{Improving ELIZA using feedback}
\author{Ian McLoughlin (ian.mcloughlin@gmit.ie)}
\date{\today}

\geometry{
 a4paper,
 total={170mm,257mm},
 left=30mm,
 right=30mm,
 top=10mm,
 }

% Everything above the below \begin{document} command is referred to as the
% preamble. The preamble configures the features and styles of the output.
% The \begin{document} command tells latex we beginning the content of the
% document - the words, sentences, and paragraphs we want to create.

% Begin the document.  
\begin{document}

  % Display the title in the standard way for the document class.
  \maketitle

  % Usually, the first item after the title is an abstact.
  % An abstract gives a short summary of the main contribution of the document.
  \begin{abstract}
    We propose to research and develop an impovement to Joseph Weisenbaum's
    ELIZA program. The ELIZA program simulates a psychiatrist's questions and
    responses to a patient. The user of the program is the patient in the
    simulation. The program was designed to demonstrate much of the
    superficiality of human conversation. The original program used
    reflective text-based substitutions. Carefully crafted user inputs could
    easily demonstrate that they were talking to a simulation rather than a
    person. We are proposing to use regular expressions to further enhance the
    program, making it somewhat more difficult to demonstrate it is a
    simulation.
  \end{abstract}

  % We generally organise our document then in sections. A section is the
  % highest level of organisation, then subsection, then subsubsection. The
  % second subsection of the third section would be listed as part 3.2 of the
  % document. You can either use the command \section{My section} to begin a 
  % new section or you can use \begin{section}{My section} and \end{section} to
  % the same effect. The former is generally preferred as it's cleaner and latex
  % can generally figure out where the section ends.
  \section{Introduction}
    The introduction generally overlaps with the abstract, but expands greatly
    on the ideas. It also gives an overview of the document, giving a one or 
    two sentence overview of each part of the document.

    You can start a new paragraph by leaving a blank line. This is a new
    paragraph.
    % References can be inserted using the \cite{} command with the item's
    % bibtex name in the curly braces. This will insert the reference indicator
    % where the \cite command is called and also ensure the reference title,
    % authours, etc in the list of references. The ~ controls the amount of
    % whitespace around the reference, try googling "latex reference tilde".
    ELIZA is a program written by Joseph Wiesenbaum. It was proposed in his 1966
    paper~\cite{weisenbaumeliza}.

  % Let's create a subsection.
  \subsection{Hello, subsection}
    This is a subsection. It contains text.

  % Second section.
  \section{Literature}
    % We can give most high-level items a label with the \label command.
    % We'll come back to that in a minute.
    \label{section:literature}
    Typically a research proposal will give a bit of background, listing four or
    five directly relevant peer-reviewed publications.


  \section{Timeline}
  \begin{figure}[H]
    \begin{center}
      \begin{ganttchart}[title/.style={draw=none},vgrid,hgrid,canvas/.append style={draw=gmitgrey},bar/.append style={fill=gmitblue}]{1}{24}
        %labels
        %\gantttitle{Project}{24}\\
        \gantttitle{Q4'20}{3}
        \gantttitle{Q1'21}{3}
        \gantttitle{Q2'21}{3}
        \gantttitle{Q3'21}{3}
        \gantttitle{Q4'21}{3}
        \gantttitle{Q1'22}{3}
        \gantttitle{Q2'22}{3}
        \gantttitle{Q3'22}{3} \\
        %tasks
        \ganttbar{Literature}{1}{3} \\
        \ganttbar{Develop}{3}{12} \\
        \ganttbar{Test}{12}{15} \\
        \ganttbar{Finalise}{15}{24} \\
      \end{ganttchart}
    \end{center}
    \caption{Gantt Chart}
  \end{figure}

  % Latex together with bibtex will generate a list of references from the
  % \cite commands in the latex document and the items in the bib file.

  % The style of referencing is controlled by the \bibliographystyle command.
  % There are loads of styles available - google "bibtex styles".
  % The ieeetr style uses numbers for references which is common in computing.
  \bibliographystyle{ieeetr}
  % The \bibliography command tells latex where the bib file is. The bib file
  % is called "bibliography.bib" in the current folder. You omit the ".bib".
  \bibliography{bibliography}

% The end of the document.
\end{document}
